\documentclass[a4paper]{article}

%very bad name to change quickly
\newcommand{\spec}{\emph{RobotMove}\ }

\newcommand{\rhoban}{\emph{Rhoban}\ }
\newcommand{\rhobanp}{\emph{Rhoban ARM9 Plateform}\ }
\newcommand{\version}{20101220\ }

\newcommand{\instruction}[1]{{\tt #1}}


\newcommand{\file}[1]{\emph{#1}}

\title{A robotic move specification language}

\author{Hugo Gimbert \and Olivier Ly}

\begin{document}
\maketitle

\begin{abstract}
This document describes the move specification language
\spec used
in the \rhoban project. 
\end{abstract}

\section{Miscellaneous ideas}

An efficient robot move controller requires:
\begin{enumerate}
  \item reference target trajectories to guide the move (target splines)
  \item interpretation of sensors values as errors (compute distance to splines)
  \item real-time corrections based on errors (fit the splines)
  \item state-based emergency transitions (horse fall management)
  \item state-based regular transitions (finite automata and control splines)
  \item parallel composition of moves
  \item hierarchical composition of moves, with variable time scales
\end{enumerate}

 

\section{Syntax}

\section{Semantic}

\section{Embedded Scheduling}

\end{document}
