%%This is a very basic article template.
%%There is just one section and two subsections.

\documentclass[a4paper]{article}


\newcommand{\rhoban}{\emph{Rhoban}\ }
\newcommand{\rhobanp}{\emph{Rhoban ARM9 Plateform}\ }
\newcommand{\version}{20101220\ }

\newcommand{\instruction}[1]{{\tt #1}}


\newcommand{\file}[1]{\emph{#1}}

\title{Using the \rhobanp}

\author{Hugo Gimbert \and Olivier Ly}

\begin{document}
\maketitle

\section{Introduction}

The \rhobanp is a set of tools which provide an easy way to
quickly design and control robotic prototypes.
The \rhobanp is suitable for prototypes driven by AT91
micro-controllers running Linux.

\subsection{Features}

The main feature of the \rhobanp is to
provide easy access to and control of devices physically connected to the
AT91 micro-controller. In particular, the \rhobanp allows:
\begin{itemize}
  \item controlling generic digital output signals,
  \item sampling digital input signals,
  \item controlling servos and motors via pwm signals,
  \item controlling Dynamixel servos,
  \item sampling analog sensors,
  \item sampling I2C sensors,
  \item controlling cameras,
  \item controlling I2C and serial devices,
  \item etc\ldots
\end{itemize}

\subsection{Content}

The \rhobanp consists in:
\begin{itemize}
  \item a module which takes care of control and communication with
  the physical devices connected to an AT91 micro-controller,
   \item a C API which allows easy access to the devices from userland,
   \item a tool chain to compile C/C++ embedded programs,
   \item documentation,
   \item technical support from the \rhoban team ;-).
\end{itemize}


\subsection{License}

The \rhobanp was developped by Hugo Gimbert and Olivier Ly
and is the property of its developpers.

Although the \rhobanp is proprietary software, it can be used
freely for recreational and educational purposes.

Any commercial use is disallowed unless and end-user license agreement
has been granted to the user for this particular use.

\section{Quick Start Guide}

Before starting working with the \rhobanp, the user should get familiar with
Linux and C/C++ programming under Linux. This guide is applicable to
version \version of the \rhobanp.
Version \version is based on Linux kernel 2.6.


\subsection{Using the C API control\_low\_level}

This section provides a rough description of the low level API.
More details are to be found in the header file \file{control\_low\_level.h}.

\subsubsection{Initializing the low level system}

First thing to do is initializing the low level system.
This creates a communication link between the module
and the API.
This is done by a call to:

\instruction{int low\_level\_init(void);}

\subsubsection{Listing and accessing devices}

After having initialized the low level,
the list of devices can be obtained in
two ways. A pretty print of devices can be obtained using
\instruction{void print\_devices\_infos(void)}.
The complete list of devices and pointers to the devices themselves
is stored in the \instruction{extern ModuleConnexion* etat\_global} structure.

\subsubsection{Writing outputs}

A desirable basic feature is to control the state of the output pins of the
controller. For each pin to control, load a digital writer device using:

\instruction{
DigitalWriter * attach\_digital\_writer(ui32 frequency, ui32 pin, const char *
name)}
The frequency is the refresh frequency of the pin value by the module.
The pin to be driven is specified by an integer, using the encodiing in
\file{arch/arm/mach-at91/include/mach/gpio.h}. Name is an optional name to be
given to the writer, and can be set to NULL.


To actually set an output, use
\instruction{
void set\_digital\_output(ui32 id, ui32 value)
}
.

Digital writers can be used to send sequences of pulses,
which are specified by a sequence of durations and values,
see the \instruction{Pulse} and \instruction{DigitalWriter} for more details.



\subsubsection{Attaching a pwm device}

Pwm devices can be used to generate pwm signals and thus directly control
pwm servos.

To attach a pwm device use
\instruction{
PwmDevice * attach\_pwm(ui32 frequency,ui32 pin,  const char *
pwm\_name)}
where frequency is the frequency of th epwm signal (50 Hz) for most devices.
Only pins associated to a TIOA or TIOB signal can generate pwm signals this
way. Each timer of the at91 microcontroller generates two signals TIOA and
TIOB. Timer used by Linux (usually timers 0 and 1) should not be
concurrently accessed by the module.
On the at91sam9260 micro controller, pins usable to generate pwm signals are
AT91\_PIN\_PB0 (TIOA3), AT91\_PIN\_PB1 (TIOB3),  AT91\_PIN\_PB2 (TIOA4), AT91\_PIN\_PB3
(TIOA5),  AT91\_PIN\_PB18 (TIOB4) and AT91\_PIN\_PB19 (TIOB5).
 
To set the cyclic rate of the signal, the signal should be first enabled
and then the cyclic rate can be set using \instruction{void
set\_pwm\_cyclic\_rate (PwmDevice *, ui32 rate)}. The rate is expressed in
thousands, for example at 50Hz (20ms) , a rate of 80 gives rise to high state during 20/1000*80=160
$\mu$s.

\subsubsection{Reading sensors}

Sensor devices include digital reader, magnetic sensor and analogic input.
Each sensor has a refresh frequency.
With each sensor device is associated the
queue of the last 512 values measured at the sensor frequency.

\subsubsection{Reading inputs}

DigitalReaders are sensor devices that can be used to read the state of an input
pin. Actually, not only digital readers read can be used to access the current
value of a pin
but they also monitor the sequence of state changes of an input pin.

To create a digital reader, use:
\instruction{
DigitalWriter * attach\_digital\_writer(ui32 frequency, ui32 pin,
const char * name)}

\subsubsection{Sampling an analogic input}

AnalogicSensors are sensor devices that can be used to read analogic values.

To create a digital reader, use:
\instruction{
DigitalWriter * attach\_digital\_writer(ui32 frequency, ui32 pin,
const char * name)}

On the at91sam9260 micro controller, there are four pins that can be used
to measure analogic values are AT91\_PIN\_PC0 (AD0), AT91\_PIN\_PC1 (AD1),  AT91\_PIN\_PC2 (AD2)
AT91\_PIN\_PC3 (AD3).


\subsubsection{Attaching a Dynamixel system}

Also system of dynamixel servos can be controlled by the \rhobanp.




\end{document}
